\documentclass[spanish,a4paper]{article} 

\usepackage{babel}
\usepackage{soul}
\usepackage[latin1]{inputenc}
%\usepackage{bbm}
\usepackage{framed}
\input{../../../macros/Algo1Macros}




\begin{document}

\materia{Algoritmos y Estructura de Datos I}
\cuatrimestre{1}
\anio{2015}

\fecha{6 de Mayo de 2015}

\nombre{\LARGE TPI - Flores vs Vampiros}

\titulotp

\section{Introducci\'on}

En el TPE, hemos especificado el comportamiento de una serie de 
problemas relacionados con el juego Flores vs Vampiros. Nuestro pr\'oximo paso ser\'a
obtener una implementaci\'on en lenguaje imperativo, para esto se brinda una adaptaci\'on de la especificaci\'on original.

El objetivo de este trabajo es programar toda la especificaci\'on, usando los
comandos de C++ vistos en clase. Pueden utilizar las clase \texttt{vector} provista en la stl de C++, pero s\'olo se permitir\'a el uso de los m\'etodos especificados en este enunciado.

En la p\'agina de la materia estar\'an disponibles los \emph{headers} de las clases que deber\'an
implementar y la especificaci\'on de cada uno de sus m\'etodos.

Pueden agregar los m\'etodos auxiliares que necesiten. Estos deben
estar \emph{en todos los casos} en la parte privada de la clase. No se
permite modificar la parte p\'ublica de las clases ni agregar atributos
adicionales, ya sean p\'ublicos o privados. 


\section{Clase \texttt{std::vector}}
Usaremos el tipo [T] (lista) para especificar la clase vector$<$T$>$. 

\begin{problema}[this]{nuevoV}{}{[T]}
\asegura{\longitud{\res} == 0}
\end{problema}

\begin{problema}{size}{this: [T]}{\ent}
\asegura{\res == \longitud{this}}
\end{problema}

\begin{problema}{empty}{this: [T]}{\bool}
\asegura{\res == (\longitud{this} == 0)}
\end{problema}

\begin{problema}{at}{this: [T], i: \ent}{T}
\requiere{0 \leq i < \longitud{this}}
\asegura{\res == this_i}
\end{problema}
Nota: el operador [] cumple la misma especificaci\'on, tambi\'en pueden usarlo.

\begin{problema}{push\_back}{this: [T], e: T}{}
\modifica{this}
\asegura{this == pre(this) ++ [e]}
\end{problema}

\begin{problema}{pop\_back}{this: [T]}{}
\requiere{\longitud{this}>0}
\modifica{this}
\asegura{pre(this) ==this ++ [e]}
\end{problema}

\section{Implementaci\'on de tipos}

Los tipos Flor, Vampiro, Nivel y Juego mantienen, dentro de la clase, atributos que son similares a sus observadores y por lo tanto es trivial comprender su equivalencia. Dentro de la clase Nivel pueden encontrarse los structs \texttt{FlorEnJuego}, \texttt{VampiroEnJuego} y \texttt{VampiroEnEspera}, los cuales representan las triplas que contienen las flores, los vampiros y el spawning de cada nivel.

\section{Entrada/Salida}

Todas las clases del proyecto tienen tres m\'etodos relacionados con
entrada salida:

\begin{description}
\item[mostrar]: que se encarga de mostrar todo el contenido de la
  instancia de la clase en el flujo de salida indicado. El formato
  es a gusto del consumidor, pero esperamos que sea algo m\'as o menos informativo.
\item[guardar]: que se encarga de escribir la informaci\'on de cada
  instancia en un formato predeterminado que debe ser decodificable
  por el m\'etodo que se detalla a continuaci\'on.
\item[cargar]: que se encarga de leer e interpretar informaci\'on
  generada por el m\'etodo anterior, modificando el valor del
  par\'ametro impl\'icito para que coincida con aquel que gener\'o la
  informaci\'on.
\end{description}

En definitiva, \texttt{guardar} se usar\'a para ``grabar'' en un
archivo de texto el contenido de una instancia mientras que ``cargar''
se usar\'a para ``recuperar'' dicha informaci\'on. En todos los casos, sus
interfaces ser\'an:

\begin{itemize}
\item\verb|void mostrar(std::ostream& ) const|;
\item\verb|void guardar(std::ostream& ) const|;
\item\verb|void cargar(std::istream& )|;
\end{itemize}

El detalle del formato que se debe usar para ``guardar'' y ``leer'' en
cada clase se indica a continuaci\'on. Tener en cuenta que los n\'umeros
se deben guardar como texto. Es decir, si escriben a un archivo y
despu\'es lo miran con un editor de texto, donde escribieron un n\'umero
65 deben ver escrito 65 y no una letra A.


\begin{description}
\item [Flor:] Se debe guardar \verb|"{ F "|, su vida, \verb|" "|,  su cuantoPega, \verb|" "|, su lista de habilidades separadas por espacio y \verb|" }"|. Por ejemplo, una flor que pueda Atacar y Explotar se guarda:
\begin{verbatim}
{ F 33 6 [ Atacar Explotar ] } 
\end{verbatim}

\item [Vampiro:] Se debe guardar \verb|"{ V "|, su clase de vampiro, \verb|" "|, su vida, \verb|" "|, su cuanto pega y \verb|" }"|. Por ejemplo, un vampiro de clase Desviado con 50 de vida y 8 de cuantoPega se guarda:
\begin{verbatim}
{ V Desviado 50 8 }
\end{verbatim}


\item [Nivel:] Se debe guardar \verb|"{ N "|, su ancho, \verb|" "|, su alto, \verb|" "|, su turno, \verb|" "|, sus soles,  \verb|" "|, su lista de flores en el juego, \verb|" "|, su lista de vampiros en el juego,  \verb|" "|, su lista de spawning y \verb|" }"|. 
A continuaci\'on se muestra el ejemplo de un nivel al quedar guardado:

\begin{verbatim}
{ N 5 5 3 56 
  [ ( { F 33 6 [ Atacar Explotar ] } ( 2 3 ) 10 ) ( { F 33 0 [ Generar Explotar ] } ( 2 5 ) 33 ) ]
  [ ( { V Desviado 50 8 } ( 4 5 ) 10 ) ] 
  [ ( { V Caminante 30 7 } 2 5 ) ] } 
\end{verbatim}


\item [Juego:] Se debe guardar \verb|"{ J "|, su lista de flores, \verb|" "|, su lista de vampiros, , \verb|" "|, su lista de niveles y \verb|" }"|.
A continuaci\'on se muestra el ejemplo de un juego al quedar guardado:
\begin{verbatim}
{ J [ { F 33 6 [ Atacar Explotar ] } { F 33 0 [ Generar Explotar ] } ]
    [ { V Desviado 50 8 } { V Caminante 30 7 } ]
    [ { N 5 5 3 56 
      [ ( { F 33 6 [ Atacar Explotar ] } ( 2 3 ) 10 ) ( { F 33 0 [ Generar Explotar ] } ( 2 5 ) 33 ) ]
      [ ( { V Desviado 50 8 } ( 4 5 ) 10 ) ] 
      [ ( { V Caminante 30 7 } 2 5 ) ] } ] }
\end{verbatim}



Aclaraci\'on: Tanto las listas, como las tuplas, tienen todas sus componentes separadas por espacios. Adem\'as los saltos de linea en los ejemplos de nivel y juego son simplemente para la presentaci\'on en este documento.

\end{description}



\end{document}
